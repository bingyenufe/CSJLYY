\documentclass[]{ctexbook}
\usepackage{lmodern}
\usepackage{amssymb,amsmath}
\usepackage{ifxetex,ifluatex}
\usepackage{fixltx2e} % provides \textsubscript
\ifnum 0\ifxetex 1\fi\ifluatex 1\fi=0 % if pdftex
  \usepackage[T1]{fontenc}
  \usepackage[utf8]{inputenc}
\else % if luatex or xelatex
  \ifxetex
    \usepackage{xltxtra,xunicode}
  \else
    \usepackage{fontspec}
  \fi
  \defaultfontfeatures{Ligatures=TeX,Scale=MatchLowercase}
\fi
% use upquote if available, for straight quotes in verbatim environments
\IfFileExists{upquote.sty}{\usepackage{upquote}}{}
% use microtype if available
\IfFileExists{microtype.sty}{%
\usepackage{microtype}
\UseMicrotypeSet[protrusion]{basicmath} % disable protrusion for tt fonts
}{}
\usepackage[b5paper,tmargin=2.5cm,bmargin=2.5cm,lmargin=3.5cm,rmargin=2.5cm]{geometry}
\usepackage[unicode=true]{hyperref}
\PassOptionsToPackage{usenames,dvipsnames}{color} % color is loaded by hyperref
\hypersetup{
            pdftitle={财税计量应用},
            pdfauthor={叶兵},
            colorlinks=true,
            linkcolor=Maroon,
            citecolor=Blue,
            urlcolor=Blue,
            breaklinks=true}
\urlstyle{same}  % don't use monospace font for urls
\usepackage{natbib}
\bibliographystyle{apalike}
\usepackage{longtable,booktabs}
% Fix footnotes in tables (requires footnote package)
\IfFileExists{footnote.sty}{\usepackage{footnote}\makesavenoteenv{long table}}{}
\IfFileExists{parskip.sty}{%
\usepackage{parskip}
}{% else
\setlength{\parindent}{0pt}
\setlength{\parskip}{6pt plus 2pt minus 1pt}
}
\setlength{\emergencystretch}{3em}  % prevent overfull lines
\providecommand{\tightlist}{%
  \setlength{\itemsep}{0pt}\setlength{\parskip}{0pt}}
\setcounter{secnumdepth}{5}
% Redefines (sub)paragraphs to behave more like sections
\ifx\paragraph\undefined\else
\let\oldparagraph\paragraph
\renewcommand{\paragraph}[1]{\oldparagraph{#1}\mbox{}}
\fi
\ifx\subparagraph\undefined\else
\let\oldsubparagraph\subparagraph
\renewcommand{\subparagraph}[1]{\oldsubparagraph{#1}\mbox{}}
\fi

% set default figure placement to htbp
\makeatletter
\def\fps@figure{htbp}
\makeatother

\usepackage{booktabs}
\usepackage{longtable}

\usepackage{framed,color}
\definecolor{shadecolor}{RGB}{248,248,248}

\renewcommand{\textfraction}{0.05}
\renewcommand{\topfraction}{0.8}
\renewcommand{\bottomfraction}{0.8}
\renewcommand{\floatpagefraction}{0.75}

\let\oldhref\href
\renewcommand{\href}[2]{#2\footnote{\url{#1}}}

\makeatletter
\newenvironment{kframe}{%
\medskip{}
\setlength{\fboxsep}{.8em}
 \def\at@end@of@kframe{}%
 \ifinner\ifhmode%
  \def\at@end@of@kframe{\end{minipage}}%
  \begin{minipage}{\columnwidth}%
 \fi\fi%
 \def\FrameCommand##1{\hskip\@totalleftmargin \hskip-\fboxsep
 \colorbox{shadecolor}{##1}\hskip-\fboxsep
     % There is no \\@totalrightmargin, so:
     \hskip-\linewidth \hskip-\@totalleftmargin \hskip\columnwidth}%
 \MakeFramed {\advance\hsize-\width
   \@totalleftmargin\z@ \linewidth\hsize
   \@setminipage}}%
 {\par\unskip\endMakeFramed%
 \at@end@of@kframe}
\makeatother

\makeatletter
\@ifundefined{Shaded}{
}{\renewenvironment{Shaded}{\begin{kframe}}{\end{kframe}}}
\@ifpackageloaded{fancyvrb}{%
  % https://github.com/CTeX-org/ctex-kit/issues/331
  \RecustomVerbatimEnvironment{Highlighting}{Verbatim}{commandchars=\\\{\},formatcom=\xeCJKVerbAddon}%
}{}
\makeatother

\usepackage{makeidx}
\makeindex

\urlstyle{tt}

\usepackage{amsthm}
\makeatletter
\def\thm@space@setup{%
  \thm@preskip=8pt plus 2pt minus 4pt
  \thm@postskip=\thm@preskip
}
\makeatother

\frontmatter

\title{财税计量应用}
\author{叶兵}
\date{2022-04-22}

\begin{document}
\maketitle


\thispagestyle{empty}

\begin{center}
献给……

呃,爱谁谁吧
\end{center}

\setlength{\abovedisplayskip}{-5pt}
\setlength{\abovedisplayshortskip}{-5pt}

{
\setcounter{tocdepth}{2}
\tableofcontents
}
\listoftables
\listoffigures
\hypertarget{ux5e8fux8a00}{%
\chapter*{序言}\label{ux5e8fux8a00}}


\textbf{本书的写作动机}

国内本科计量经济学教学和研究严重脱节。这不是说研究高深莫测,超出了本科生的理解范围。实际上,计量经济学研究用到的主流的方法并不难理解。
市面上不乏好的计量经济学教材,如伍德里奇(2015)、斯托克和沃森(2012)、安格里斯特和皮施克(2012)。但这些书内容芜杂,涉及到过多的数学,侧重理论,涉及应用较少。
实有必要写这样一本书:内容可控,数学尽量简单,本科生容易理解,学习后能读懂大部分应用计量文章,并能开展简单的研究。这是本书的目的。
本书最贴合的目标对象是公共经济学的本科生,因为书中举的大部分例子属于公共经济学领域。但由于经济学各子学科所用计量方法实无太大的差异,本书也能满足经济学大类本科生的要求。

\textbf{前置知识}

本书的前置知识括《概率论与数理统计》《统计学》《财政学》《公共支出管理》的知识。本书也会对涉及到的知识做简要回顾。

\textbf{本书的内容}

各章目次如下:

\begin{longtable}[]{@{}ll@{}}
\toprule
章序号 & 章名 \\
\midrule
\endhead
第一章 & 导论 \\
第二章 & 二元回归模型 \\
第三章 & 多元回归分析 \\
第四章 & 定性信息、清晰断点回归、一致性、机制和异质性分析 \\
第五章 & 面板数据模型:计量方法及应用 \\
第六章 & 基于面板数据的双重差分法:计量方法及应用 \\
第七章 & 工具变量法:计量方法及应用 \\
第八章 & 科学写作 \\
第九章 & 应用:政府支出 \\
第十章 & 应用:政府收入 \\
第十一章 & 应用:政府间关系(含模糊断点回归) \\
第十二章 & 应用:其他 \\
\bottomrule
\end{longtable}

本书分为两部分。第一部分讲当前公共经济学领域应用计量研究中最流行的几种方法。在讲方法的时候穿插应用。在这部分,作者大量采取拿来主义的做法,凡在文献中已经有很好的表述的,只提供一个指向该文献的指针,以示对原作者知识产权的尊重。只有那些在文献中找不到表述,或文献中的表述不适合本科生的,才提供完整的文本。
第二部分以专题的形式讲这些计量方法的应用。
两部分涉及应用,作者都通过讲解一流英文杂志上公共经济学领域的文章展示。

\textbf{软件实现}

本书既名《财税计量应用》,就强调应用,要求学生能够利用专业的计量软件Stata估计各计量模型的结果。为此,我们录制了配套视频,循序渐进地教读者使用Stata进行相关操作。
读者还可以通过以下渠道学习Stata:

\begin{itemize}
\tightlist
\item
  Stata帮助文档
\item
  陈强. 计量经济学及Stata应用{[}M{]}. 高等教育出版社, 2015
\item
  遇到问题在网上搜索问题的关键词
\end{itemize}

\textbf{参考文献}

\begin{itemize}
\tightlist
\item
  伍德里奇. 计量经济学导论(第五版){[}M{]}. 中国人民大学出版社, 2015.
\item
  斯托克, 沃森. 计量经济学{[}J{]}. 格致出版社, 2012.
\item
  安格里斯特, 皮施克. 基本无害的计量经济学:实证研究者指南{[}M{]}. 格致出版社, 2012.
\end{itemize}

\begin{flushright}
叶兵\\
于紫东\\
\end{flushright}

\mainmatter

\hypertarget{intro}{%
\chapter{导论}\label{intro}}

图 \ref{fig:chap1fig1} 是一幅图。

\hypertarget{wind}{%
\chapter{白苹风末}\label{wind}}

瞎扯几句。

\hypertarget{ux5f20ux8001ux7237ux5b50}{%
\section{张老爷子}\label{ux5f20ux8001ux7237ux5b50}}

话说张老爷子写了一首诗:

\begin{quote}
姑苏开遍碧桃时,邂逅河阳女画师。\\
红豆江南留梦影,白苹风末唱秋词。
\end{quote}

\hypertarget{ux5f6dux5927ux5c06ux9886}{%
\section{彭大将领}\label{ux5f6dux5927ux5c06ux9886}}

貌似大家都喜欢用白萍风这个意境。又如彭玉麟的对联:

\begin{quote}
凭栏看云影波光,最好是红蓼花疏、白苹秋老;\\
把酒对琼楼玉宇,莫孤负天心月到、水面风来。
\end{quote}

嘿,玛尼玛尼哄。

\hypertarget{ux5f6dux5927}{%
\section{彭大}\label{ux5f6dux5927}}

\cleardoublepage

\hypertarget{appendix-ux9644ux5f55}{%
\appendix \addcontentsline{toc}{chapter}{\appendixname}}


\hypertarget{sound}{%
\chapter{余音绕梁}\label{sound}}

呐,到这里朕的书差不多写完了,但还有几句话要交待,所以开个附录,再啰嗦几句,各位客官稍安勿躁、扶稳坐好。

\bibliography{book.bib,packages.bib}

\backmatter
\printindex

\end{document}
